%
%  This is rather horrible LaTeX-code as there are plenty of explicit
% \vspace*{...}. This became necessary as there is not much maneuvering
% space on a double sided leaflet for so many tables as we need here.
%
% 04/05-MAY-2016    Bernd Ulmann
%
\documentclass{leaflet}
\usepackage{longtable}
\begin{document}
 \title{\vspace*{-1cm}QNICE programming card\\~\\ISA v1.7}
 \maketitle
 \vspace*{-1cm}
%
 \section{General}
  QNICE features 16 bit words, 16 registers, 4 addressing modes, and 
  a 16 bit address space (16 bit words, upper 256 words reserved for
  memory mapped I/O).
  \vspace*{-5mm}
%
 \section{Registers}
  All in all there are 16 general purpose registers (\emph{GPR}s) available:
  \begin{center}
   \vspace*{-2mm}
   \begin{longtable}{|c|c|c||c|c|c|c|c|}
    \hline
    {\tt R0}&\dots&{\tt R7}&{\tt R8}&\dots&{\tt R13}&{\tt R14}&{\tt R15}\\
    \hline
   \end{longtable}
   \vspace*{-9mm}
  \end{center}
  \begin{description}
   \item [\texttt{R0}\dots\texttt{R7}:] GPRs, actually 
    these are a window into a register bank holding $256\times 8$ such 
    registers.
   \item [\texttt{R13}:] Stack pointer (\texttt{SP}).
   \item [\texttt{R14}:] Statusregister (\texttt{SR}).
   \item [\texttt{R15}:] Program counter (\texttt{PC}).
  \end{description}
%
  \subsection{Statusregister}
   \begin{center}
    \begin{longtable}{|c||c|c|c|c|c|c|c|c|}
     \hline
     {\tt rbank}&
     ---&---&{\tt V}&{\tt N}&{\tt Z}&{\tt C}&{\tt X}&{\tt 1}\\
     \hline
    \end{longtable}
    \vspace*{-9mm}
   \end{center}
   \begin{description}
    \item [{\tt 1}:] Always set to 1.
    \item [{\tt X}:] Used by shift instructions only.
    \item [{\tt C}:] Carry flag.
    \item [{\tt Z}:] 1 if the last result was {\tt 0x0000}.
    \item [{\tt N}:] 1 if the last result was negative.
    \item [{\tt V}:] 1 if the last operation caused an overflow, i.\,e. two
     positive operands yielded a negative result or vice versa.
   \end{description}
   The upper eight bits of \texttt{SR} hold the pointer to the register
   window. Changing the value stored here will yield a different set of 
   GPRs \texttt{R0}\dots\texttt{R7} which is especially useful for subroutine
   calls (a stack of registers, so to speak).
   \vspace*{-5mm}
%
 \section{Instruction Set}
  QNICE features 14 basic instructions, four jump/branch instructions, three
  control instructions, and four adressing modes.
  \vspace*{-3mm}
  {\scriptsize
   \begin{center}
    \begin{longtable}{|c|ll|l|}
     \hline
      $\!\!\!$Opc$\!\!\!$&Instr&Operands&Effect\\
     \hline
     \hline
      {\tt 0}&{\tt MOVE}&{\tt src, dst}&{\tt dst := src}\\
      {\tt 1}&{\tt ADD}&{\tt src, dst}&{\tt dst := dst + src}\\
      {\tt 2}&{\tt ADDC}&{\tt src, dst}&{\tt dst := dst + src + C}\\
      {\tt 3}&{\tt SUB}&{\tt src, dst}&{\tt dst := dst - src}\\
      {\tt 4}&{\tt SUBC}&{\tt src, dst}&{\tt dst := dst - src - C}\\
      {\tt 5}&{\tt SHL}&{\tt src, dst}&{\tt dst << src}, fill with X, shift to C\\
      {\tt 6}&{\tt SHR}&{\tt src, dst}&{\tt dst >> src}, fill with C, shift to X\\
      {\tt 7}&{\tt SWAP}&{\tt src, dst}&{\tt dst := ((src << 8) \& 0xFF00) |}\\
             &          &              &~~~~~~~~~~{\tt ((src >> 8) \& 0xFF)}\\
      {\tt 8}&{\tt NOT}&{\tt src, dst}&{\tt dst := !src}\\
      {\tt 9}&{\tt AND}&{\tt src, dst}&{\tt dst := dst \& src}\\
      {\tt A}&{\tt OR}&{\tt src, dst}&{\tt dst := dst | src}\\
      {\tt B}&{\tt XOR}&{\tt src, dst}&{\tt dst := dst \^\ src}\\
      {\tt C}&{\tt CMP}&{\tt src, dst}&compare {\tt src} with {\tt dst}\\
      {\tt D}&reserved&&\\
      {\tt E}&{\tt HALT}&&Halt the processor\\
      {\tt E}&{\tt RTI}&&Return from interrupt\\
      {\tt E}&{\tt INT}&dst&Issue software interrupt\\
      {\tt E}&{\tt INCRB}&&Increment register bank address\\
      {\tt E}&{\tt DECRB}&&Decrement register bank address\\
      {\tt E}&{\tt EXC}&{\tt const, dst}&Exchange shadow register\\
      {\tt F}&{\tt ABRA}&{\tt dst, [!]cond}&Absolute branch\\
      {\tt F}&{\tt ASUB}&{\tt dst, [!]cond}&Absolut subroutine call\\
      {\tt F}&{\tt RBRA}&{\tt dst, [!]cond}&Relative branch\\
      {\tt F}&{\tt RSUB}&{\tt dst, [!]cond}&Relative subroutine call\\
     \hline
    \end{longtable}
   \end{center}
  }
%
  \subsection{Basic Instructions (opcodes \texttt{0}..\texttt{C})}
   \begin{center}
    \begin{longtable}{|c||c|c||c|c|}
     \hline
     4 bit&4 bit&2 bit&4 bit&2 bit\\
     {\tt opcode}&{\tt src rxx}&{\tt src mode}&
     {\tt dst rxx}&{\tt dst mode}\\
     \hline
    \end{longtable}
   \end{center}
  \vspace*{-14mm}
%
  \subsection{Control instructions}
    \begin{center}
     \begin{longtable}{|c||c||c|c|}
      \hline
      4 bit&6 bit&4 bit&2 bit\\
      {\tt opcode=E}&command&\texttt{dst rxx}&\texttt{dst mode}\\
      \hline
     \end{longtable}
    \end{center}
   \vspace*{-13mm}
%
  \subsection{Jumps and Branches}
   {\scriptsize
    \begin{center}
     \begin{longtable}{|c||c|c||c||c|c|}
      \hline
      4 bit&4 bit&2 bit&2 bit&1 bit&3 bit\\
      &     &     &     &{\tt negate}&{\tt select}\\
      {\tt opcode=F}&{\tt src rxx}&{\tt src mode}&
      {\tt mode}&{\tt condition}&{\tt condition}\\
      \hline
     \end{longtable}
    \end{center}
   }
   \vspace*{-13mm}
%
  \subsection{CMP}
   The CMP (compare) instruction can be used for signed as well as for unsigned
   comparisons:
   \begin{center}
    \vspace*{-2mm}
    \begin{longtable}{|l|l|l|l|l|}
     \hline
     Condition&\multicolumn{4}{|c|}{Flags}\\
              &\multicolumn{2}{|c|}{unsigned}&\multicolumn{2}{|c|}{signed}\\
              &\texttt{Z}&\texttt{N}&\texttt{Z}&\texttt{V}\\
     \hline
     \hline
     \texttt{src}$<$\texttt{dst}&\texttt{0}&\texttt{0}&\texttt{0}&\texttt{0}\\
     \texttt{src}$=$\texttt{dst}&\texttt{1}&\texttt{0}&\texttt{1}&\texttt{0}\\
     \texttt{src}$>$\texttt{dst}&\texttt{0}&\texttt{1}&\texttt{0}&\texttt{1}\\
     \hline
    \end{longtable}
    \vspace*{-8mm}
   \end{center}
%
  \subsection{Addressing Modes}
   {\scriptsize
    \begin{center}
     \begin{longtable}{|c|l|l|}
      \hline
       Mode bits&Notation&Description\\
      \hline
      \hline
       {\tt 00}&{\tt Rxx}&Use Rxx as operand\\
       {\tt 01}&{\tt @Rxx}&Use the memory cell addressed by\\
               &          &the contents of Rxx as operand\\
       {\tt 10}&{\tt @Rxx++}&Use the memory cell addressed by\\
               &          &the contents of Rxx as operand and\\
               &          &then increment Rxx\\
       {\tt 11}&{\tt @--Rxx}&Decrement Rxx and then use the\\
               &          &memory cell addressed by Rxx as\\
               &          &operand\\
      \hline
     \end{longtable}
    \end{center}
   }
   \vspace*{-10mm}
%
  \subsection{Shortcuts}
   The file \texttt{sysdef.asm} (part of the monitor) defines some shortcuts
   which facilitate write- and readability of QNICE assembler code:
   \vspace*{-3mm}
   \begin{center}
    \begin{longtable}{|l|l|}
     \hline
      Shortcut&Implementation\\
     \hline
     \hline
      \texttt{RET}&\texttt{MOVE @R13++, R15}\\
      \texttt{NOP}&\texttt{ABRA R15, 1}\\
      \texttt{SYSCALL(x, y)}&\texttt{ASUB x, y}\\
     \hline
      \texttt{SP}&\texttt{R13}\\
      \texttt{SR}&\texttt{R14}\\
      \texttt{PC}&\texttt{R15}\\
     \hline
    \end{longtable}
   \end{center}
   \vspace*{-15mm}
%
 \section{Interrupts}
  In case of a hardware interrupt, the processor expects the address of the 
  interrupt service routine (ISR) on the data bus. In case of a software
  interrupt (\texttt{INT}-instruction) the ISR address is specified by the 
  \texttt{dst} part of the instruction. When an interrupt occurs, the CPU
  saves the contents of \texttt{R8} to \texttt{R15} in eight shadow registers
  which can be accessed with the \texttt{EXC} instruction. An ISR must be 
  left with the \texttt{RTI}-instruction. Interrupts can not be nested.
%
 \section{Input/Output}
  I/O devices are memory mapped, their respective control and data registers 
  occupy the topmost 256 words of memory.
  {\scriptsize
   \begin{center}
    \begin{longtable}{|l|l|l|}
     \hline
     Label&Address&Description\\
     \hline
     \hline
     \texttt{IO\$BASE}&\texttt{0xFF00}&Start of I/O area\\
     \hline
     \texttt{IO\$SWITCH\_REG}&\texttt{0xFF00}&Switch register\\
     \texttt{IO\$TIL\_DISPLAY}&\texttt{0xFF01}&TIL-display\\
     \texttt{IO\$TIL\_MASK}&\texttt{0xFF02}&Mask register\\
     \texttt{IO\$KBD\_STATE}&\texttt{0xFF03}&USB-keyboard state\\
     \texttt{IO\$KBD\_DATA}&\texttt{0xFF04}&USB-keyboard data\\
     \hline
     \texttt{IO\$CYC\_LO}&\texttt{0xFF08}&Cycle counter low\\
     \texttt{IO\$CYC\_MID}&\texttt{0xFF09}&Cycle counter middle\\
     \texttt{IO\$CYC\_HI}&\texttt{0xFF0A}&Cycle counter high\\
     \texttt{IO\$CYC\_STATE}&\texttt{0xFF0B}&Cycle counter status\\
     \texttt{IO\$INS\_LO}&\texttt{0xFF0C}&Cycle counter low\\
     \texttt{IO\$INS\_MID}&\texttt{0xFF0D}&Cycle counter middle\\
     \texttt{IO\$INS\_HI}&\texttt{0xFF0E}&Cycle counter high\\
     \texttt{IO\$INS\_STATE}&\texttt{0xFF0F}&Cycle counter status\\
     \hline
     \texttt{IO\$UART\_MR1x}&\texttt{0xFF10}&UART status register\\
     \texttt{IO\$UART\_SRA}&\texttt{0xFF11}&UART status register\\
     \texttt{IO\$UART\_RHRA}&\texttt{0xFF12}&UART receive register\\
     \texttt{IO\$UART\_THRA}&\texttt{0xFF13}&UART receive register\\
     \hline
     \texttt{IO\$EAE\_OPERAND\_0}&\texttt{0xFF18}&EAE 1st operand\\
     \texttt{IO\$EAE\_OPERAND\_1}&\texttt{0xFF19}&EAE 2nd operand\\
     \texttt{IO\$EAE\_RESULT\_LO}&\texttt{0xFF1A}&EAE low result\\
     \texttt{IO\$EAE\_RESULT\_HI}&\texttt{0xFF1B}&EAE high result\\
     \texttt{IO\$EAE\_CSR}&\texttt{0xFF1C}&EAE command \& status reg.\\
     \hline
     \texttt{IO\$SD\_ADDR\_LO}&\texttt{0xFF20}&SD card low addr.\\
     \texttt{IO\$SD\_ADDR\_HI}&\texttt{0xFF21}&SD card high addr.\\
     \texttt{IO\$SD\_ADDR\_POS}&\texttt{0xFF22}&Ptr. to 512 byte bfr.\\
     \texttt{IO\$SD\_DATA}&\texttt{0xFF23}&Byte in 512 byte bfr.\\
     \texttt{IO\$SD\_ERROR}&\texttt{0xFF24}&Error code\\
     \texttt{IO\$SD\_CSR}&\texttt{0xFF25}&Command and status reg.\\
     \hline
     \texttt{IO\$TIMER\_0\_PRE}&\texttt{0xFF28}&Timer 0 prescaler\\
     \texttt{IO\$TIMER\_0\_CNT}&\texttt{0xFF28}&Timer 0 counter\\
     \texttt{IO\$TIMER\_0\_INT}&\texttt{0xFF28}&Timer 0 ISR addr.\\
     \texttt{IO\$TIMER\_1\_PRE}&\texttt{0xFF28}&Timer 1 prescaler\\
     \texttt{IO\$TIMER\_1\_CNT}&\texttt{0xFF28}&Timer 1 counter\\
     \texttt{IO\$TIMER\_1\_INT}&\texttt{0xFF28}&Timer 1 ISR addr.\\
     \hline
     \texttt{VGA\$STATE}&\texttt{0xFF30}&VGA status register\\
     \texttt{VGA\$CR\_X}&\texttt{0xFF31}&Cursor X-position\\
     \texttt{VGA\$CR\_Y}&\texttt{0xFF32}&Cursor y-position\\
     \texttt{VGA\$CHAR}&\texttt{0xFF33}&Character code\\
     \texttt{VGA\$OFFS\_DISPLAY}&\texttt{0xFF34}&Display RAM offset\\
     \texttt{VGA\$OFFS\_RW}&\texttt{0xFF35}&R/W RAM offset\\
     \texttt{VGA\$HDMI\_H\_MIN}&\texttt{0xFF36}&HDMI min. valid col.\\
     \texttt{VGA\$HDMI\_H\_MAX}&\texttt{0xFF36}&HDMI max. valid col.\\
     \texttt{VGA\$HDMI\_V\_MAX}&\texttt{0xFF36}&HDMI max. valid row\\
     \hline
    \end{longtable}
   \end{center}
  }
%
%\pagebreak%LAYOUT
  \subsection{VGA Controller}
   \subsubsection{\texttt{VGA\$STATE Bits}}
~
\vspace*{-4mm}
    \begin{center}
     \begin{longtable}{|l|l|}
      \hline
      Bit&Description\\
      \hline
      \hline
      11&Enable R/W offset register if set.\\
      10&Enable display offset register if set.\\
      9&Busy (wait for 0 before issuing command).\\
      8&Clear screen (set until completion).\\
      7&Enable VGA controller.\\
      6&Enable hardware cursor.\\
      5&Enable hardware cursor blinking.\\
      4&Hardware cursor mode:\\
       &Small if set, large if cleared.\\
      2\dots 0&Display color (RGB).\\
      \hline
     \end{longtable}
    \end{center}
\vspace*{-15mm}
   \subsubsection{\texttt{VGA\$CR\_X}}
    X coordinate of next char to be displayed.
\vspace*{-8mm}
   \subsubsection{\texttt{VGA\$CR\_Y}}
    Y coordinate of next char to be displayed.
\vspace*{-8mm}
   \subsubsection{\texttt{VGA\$CHAR}}
    Writing a byte to this register causes it to be displayed on the current
    X/Y coordinate on the screen. Reading from this register yields the 
    character at the current display coordinate.
\vspace*{-5mm}
   \subsubsection{\texttt{VGA\$OFFS\_DISPLAY}}
    This register holds the offset in bytes that is to be used when displaying
    the video RAM. To scroll one line forward, simply add \texttt{0x0050} to 
    this register. For this to work, bit 10 in \texttt{VGA\$STATE} has to be
    set.
\vspace*{-5mm}
   \subsubsection{\texttt{VGA\$OFFS\_RW}}
    Similar to \texttt{VGA\$OFFS\_DISPLAY} -- controls the offset for 
    read/write accesses to the display memory.
  \subsection{USB-Keyboard}
   \subsubsection{\texttt{IO\$KBD\_STATE}}
    \begin{center}
     \vspace*{-3mm}
     \begin{longtable}{|l|l|}
      \hline
       Bit&Description\\
      \hline
      \hline
       0&Set if an unread character is available.\\
       1&Key pressed (val in bits \texttt{15..8}\\
       2\dots 4&Keyboard layout:\\
        &\texttt{000}: US keyboard\\
        &\texttt{001}: German keyboard\\
       5\dots 7&Key modifier bit mask:\\
        &5: shift, 6: alt, 7: ctrl\\
      \hline
     \end{longtable}
    \end{center}
    \vspace*{-14mm}
   \subsection{Cycle Counter}
~\vspace*{-10mm}
    \subsubsection{\texttt{CYC\$STATE}}
~\vspace*{-3mm}
     \begin{center}
      \begin{longtable}{|l|l|}
       \hline
        Bit&Description\\
       \hline
       \hline
        0&Reset counter and start counting.\\
        1&1: count, 0: inhibit\\
       \hline
      \end{longtable}
     \end{center}
     \vspace*{-14mm}
%
  \subsection{EAE}
   \subsection{\texttt{IO\$EAE\_CSR}}
    \begin{center}
     \begin{longtable}{|l|l|}
      \hline
       Bit&Description\\
      \hline
      \hline
       0/1&Operation (MULU, MULS, DIVU, DIVS)\\
       15&Busy if set\\
      \hline
     \end{longtable}
    \end{center}
%
~\vspace*{-20mm}
  \subsection{UART}
   \subsubsection{\texttt{IO\$UART\_SRA}}
    \begin{center}
     \begin{longtable}{|l|l|}
      \hline
       Bit&Description\\
      \hline
      \hline
       0&Character received.\\
       1&Transmitter ready for next character.\\
      \hline
     \end{longtable}
    \end{center}
    \vspace*{-16mm}
%
 \section{Code Examples}
  \subsection{Typical Subroutine Call}
   \begin{verbatim}
        MOVE ..., R8     ; Setup parameters 
        ...
        RSUB SUBR, 1     ; Call subroutine
        ...
SUBR:   INCRB            ; Get free reg. set
        ...
        DECRB            ; Restore reg. set
        MOVE @R13++, R15 ; RET
   \end{verbatim}
\vspace*{-5mm}
  \subsection{Compute $\sum_{i=0}^{16}{\texttt{0x0010}}$}
   \begin{verbatim}
        .ORG 0x8000
        XOR R0,      R0 ; Clear R0
        MOVE 0x0010, R1 ; Upper limit
LOOP:   ADD R1,      R0 ; One summation
        SUB 0x0001,  R1 ; Decrement i
        ABRA LOOP,   !Z ; Loop if not zero
        HALT
   \end{verbatim}
\end{document}
